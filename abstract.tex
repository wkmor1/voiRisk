
\subsection*{Abstract}\label{abstract}
\addcontentsline{toc}{subsection}{Abstract}

Conservation and natural resource management has begun to incorporate value of information analyses into its arsenal of decision making tools. To date however, these analytical techniques have only been explored from a risk-neutral standpoint. Risk profile affects the value of information. Here, with a simple illustrative example, we demonstrate how different decision makers with different risk tolerances would value information. Intuitively one might assume that a risk-averse decision maker would always prefer to learn before making a decision. But as the example here demonstrates, this will not always be the case. Sometimes new information is less valuable to a risk-averse decision maker than it is to other, more risk-tolerant, decision makers. This is important because it highlights that not only is it important to identify critical uncertainties for decision making it is important to identify them in the context of risk-tolerence and not simply assume that risk-aversion increases information value.